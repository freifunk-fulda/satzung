\documentclass[ngerman]{article}

\usepackage[margin=2.75cm]{geometry}
\usepackage[T1]{fontenc}
\usepackage[utf8]{inputenc}
\usepackage{textcomp}
\usepackage{graphicx}
\usepackage{babel}
\usepackage{titlesec}
\usepackage{xcolor}

\titleformat{\section}{\normalfont\Large\bfseries}{\S\thesection}{1em}{}

\newcommand{\NameVerein}{Förderverein Freie Netze in Hessen}


\title{Satzung des\\
\emph{\NameVerein}}

\date{Version 0.02\\
\today}

\begin{document}

\maketitle

\thispagestyle{empty}
\pagebreak

\section{Präambel}
Die Informationsgesellschaft unserer Tage ist ohne Computer und Datennetze nicht mehr denkbar. Informations- und Kommunikationstechnologien spielen eine wichtige Rolle für den Zugang zu Bildung, Kultur und Wissenschaft. Trotz immer neuer und schnellerer digitaler Kommunikationsformen ist der Zugang zu diesen Technologien nicht allen Menschen gleichermaßen möglich. Das Streben dieser Vereinigung liegt in der Förderung eines öffentlichen, gleichberechtigten und nicht diskriminierenden Zugangs zu Informationen, unabhängig von kommerziellen Interessen. Dies wird erreicht durch den Aufbau und Betrieb einer entsprechenden Infrastruktur, sowie die Aufklärung der breiten Masse im Bezug auf die Auswirkungen freie zugänglicher Netze auf die Gesellschaft.


\section{Name, Sitz, Rechtsfähigkeit, Geschäftsjahr}
\begin{enumerate}
  \item Der Verein führt den Namen \emph{\NameVerein}.
  \item Sitz des Vereins ist Fulda.
  \item Der Verein \emph{\NameVerein} soll in das Vereinsregister des zuständigen Amtsgerichtes eingetragen werden; nach der Eintragung führt er den Zusatz \emph{e.V.}.
  \item Geschäftsjahr des Vereins ist das Kalenderjahr.
\end{enumerate}


\section{Zweck des Vereins und Gemeinnützigkeit}
\begin{enumerate}
  \item Zweck des Vereins ist die Förderung der Bildung und Kultur bezüglich kabelloser und kabelgebundener Datennetze, die der Allgemeinheit zugänglich sind, sowie die Verbreitung und Vermittlung von Wissen über Funk- und Netztechnologien.
  \item Der Satzungszweck wird insbesondere verwirklicht durch folgende Maßnahmen:
  \begin{enumerate}
    \item Information der Mitglieder und der Öffentlichkeit über freie Netze, insbesondere durch Vorträge, Veranstaltungen und Publikationen;
    \item Aufklärung über gesellschaftliche, kulturelle, gesundheitliche und rechtliche Auswirkungen freier (Funk-)netzwerke;
    \item Förderung des Zugangs zu Informationstechnologien für sozial benachteiligte Personen;
    \item Errichtung, Betrieb und Ausbau freier Datennetze;
    \item Schaffung eines modernen Datenschutzbewusstseins, unter anderem durch öffentliche Vorträge und Diskussionsrunden zu gesellschaftspolitischen Fragestellungen im Hinblick auf das Recht zur informationellen Selbstbestimmung.
  \end{enumerate}
  \item Der \emph{\NameVerein} verfolgt ausschließlich und unmittelbar gemeinnützige Zwecke im Sinne des Abschnitts "`Steuerbegünstigte Zwecke"' der Abgabenordnung (AO).
\end{enumerate}


\section{Selbstlosigkeit}
\begin{enumerate}
  \item Der Verein ist selbstlos tätig, er verfolgt nicht in erster Linie eigenwirtschaftliche Zwecke.
  \item Mittel des Vereins dürfen nur für die satzungsmäßigen Zwecke verwendet werden. Die Mitglieder erhalten keine Zuwendungen aus Mitteln des Vereins.
  \item Es darf keine Person durch Ausgaben, die dem Zweck der Körperschaft fremd sind, oder durch unverhältnismäßig hohe Vergütungen begünstigt werden.
\end{enumerate}


\section{Mitgliedschaft}
\begin{enumerate}
  \item Mitglieder des Vereins sind ordentliche Mitglieder oder Fördermitglieder. Ordentliche Mitglieder sind aktiv und in der Mitgliederversammlung stimmberechtigt.
  \item Ordentliches Mitglied des Vereins kann jede natürliche Person werden, die sich mit den Zielen des Vereins verbunden fühlt und den Verein aktiv fördern will. Bei Minderjährigen ist die Zustimmung des gesetzlichen Vertreters erforderlich. Die Mitgliedschaft ist in Textform (§126b BGB) zu beantragen. Über den Antrag entscheidet der Vorstand. Der Antrag muss den Namen und die Anschrift des Antragstellers enthalten und angeben, wie der Antragsteller den Vereinszweck aktiv fördern will.
  \item Fördermitglied kann jede natürliche Person und juristische Person jedweder Rechtsform sein, die sich mit den Zielen des Vereins verbunden fühlt und den Verein finanziell und ideell unterstützen will. Die Mitgliedschaft ist in Textform (§126b BGB) zu beantragen. Über den Antrag entscheidet der Vorstand. Der Antrag muss den Namen und die Anschrift des Antragstellers enthalten. 
  \item Gegen eine Ablehnung des Aufnahmegesuchs kann der Antragsstellende innerhalb eines Monats nach Eingang des Ablehnungsbescheids beim Vorstand schriftlich Beschwerde erheben. Die Beschwerde ist mit einer Begründung zu versehen. Über die Beschwerde entscheidet die nächste ordentliche Mitgliederversammlung.
\end{enumerate}


\section{Ende der Mitgliedschaft}
\begin{enumerate}
  \item Die Mitgliedschaft endet:
  \begin{enumerate}
    \item bei juristischen Personen mit deren Auflösung;
    \item bei natürlichen Personen mit ihrem Tod;
    \item durch freiwilligen Austritt;
    \item bei Ausschluss des Mitgliedes (§\ref{sec:ausschluss});
    \item bei Ausbleiben des Mitgliedsbeitrags länger als 6 Monate.
  \end{enumerate}
  \item Der freiwillige Austritt ist einem Vorstandsmitglied in schriftlicher Form einzureichen, die Mitgliedschaft endet 2 Wochen nach Eingang des Austrittswunsches.
  \item Das ausgetretene oder ausgeschlossene Mitglied hat keinen Anspruch gegenüber dem Vereinsvermögen.
\end{enumerate}


\section{Mitgliedsbeiträge}
\begin{enumerate}
  \item Der Verein erhebt einen Mitgliedsbeitrag, zu dessen Zahlung die Mitglieder verpflichtet sind. Näheres regelt eine Beitragsordnung, die von der Mitgliederversammlung beschlossen wird.
  \item An die Stelle der Mitgliedsbeiträge können mit Genehmigung des Vorstandes andere gleichwertige Zuwendungen treten.
\end{enumerate}


\section{Organe des Vereins}
\begin{enumerate}
  \item Die Organe des Vereins \emph{\NameVerein} sind der Vorstand und die Mitgliederversammlung.
\end{enumerate}


\section{Vorstand}
\begin{enumerate}
  \item Der Vorstand gemäß §26 BGB besteht aus den 2 Vorsitzenden und dem Kassenwart, die ordentliche Mitglieder sind und auf 1 Jahr durch die Mitgliederversammlung gewählt werden. Eine Wiederwahl ist möglich.
  \item Der Vorstand wird von der Mitgliederversammlung auf die Dauer von einem Jahr gewählt; jedes Vorstandsmitglied bleibt jedoch so lange im Amt bis eine Neuwahl erfolgt ist.
  \item Die Mitglieder des Vorstandes vertreten den Verein gerichtlich und außergerichtlich. Je 2 dieser Vorstandsmitglieder zusammen sind vertretungsberechtigt.
  \item Zu Sitzungen des Vorstandes ist eine Woche vorher schriftlich oder elektronisch zu laden. Mit dem Einverständnis aller Mitglieder des Vorstandes kann diese Frist verkürzt werden oder ganz entfallen.
  \item Der Vorstand ist beschlussfähig, wenn mehr als die Hälfte der Mitglieder des Vorstandes anwesend sind.
  \item Beschlüsse im Vorstand werden mit einfacher Mehrheit gefasst.
  \item Über Sitzungen des Vorstandes wird ein Protokoll angefertigt.
  \item Der Vorstand ist ermächtigt, gerichtlich oder behördlich geforderte Satzungsänderungen bis zur nächsten Mitgliederversammlung durchzuführen und umzusetzen.
  \item Der Kassenwart überwacht die Haushaltsführung und verwaltet das Vermögen des Vereins.
  \item Vorstandsmitglieder können jederzeit von ihrem Amt zurücktreten.
  \item Bei Rücktritt oder durch die Mitgliederversammlung festgestellte Ausübungsunfähigkeit des 1. Vorsitzenden, des 2. Vorsitzenden oder des Kassenwartes ist der gesamte Vorstand neu zu wählen. Bis zur Wahl eines neuen Vorstands ist der bisherige Vorstand zur bestmöglichen Wahrnehmung seiner Aufgaben verpflichtet.
  \item Die Vorstandsmitglieder sind grundsätzlich ehrenamtlich tätig. Sie haben Anspruch auf Erstattung notwendiger Auslagen.
\end{enumerate}


\section{Mitgliederversammlung}
\begin{enumerate}
  \item Die ordentliche Mitgliederversammlung wird einmal jährlich vom Vorstand einberufen.
  \item Eine außerordentliche Mitgliederversammlung kann vom Vorstand einberufen werden oder muss einberufen werden, wenn mindestens ein Viertel der Mitglieder die Einberufung schriftlich unter Angabe des Zwecks und der Gründe verlangt.
  \item Die Einladung zur Mitgliederversammlung ist den Mitgliedern elektronisch oder schriftlich unter Angabe von Ort, Zeit und Tagesordnung mindestens zwei Wochen im Voraus zuzustellen.
  \item Anträge von Mitgliedern, die der Tagesordnung hinzugefügt werden sollen, müssen spätestens eine Woche vor dem Termin der Mitgliederversammlung elektronisch an alle Mitglieder geschickt werden. Dazu genügt das Versenden an die Vereinsmitgliedermailingliste, in die alle Vereinsmitglieder eingetragen werden.
  \item In der Mitgliederversammlung hat jedes aktive Mitglied eine Stimme. Zur Ausübung des Stimmrechts kann ein anderes aktives Mitglied schriftlich bevollmächtigt werden. Die Bevollmächtigung ist für jede Mitgliederversammlung gesondert zu erteilen. Ein Mitglied darf jedoch nicht mehr als drei fremde Stimmen vertreten.
  \item An einer Mitgliederversammlung kann elektronisch teilgenommen werden. Form und Bedingungen der elektronischen Teilnahme werden in der Einladung bekannt gegeben.
  \item Fördermitglieder haben in der Mitgliederversammlung ein Anwesenheits- und Antragsrecht, sind aber nicht stimmberechtigt.
  \item Beschlüsse werden von der Mitgliederversammlung durch öffentliche Abstimmung getroffen. Auf Wunsch eines ordentlichen Mitglieds ist geheim abzustimmen.
  \item Über die Beschlüsse der Mitgliederversammlung ist ein Protokoll zu verfassen, das vom Versammlungsleiter und dem Schriftführer zu unterschreiben ist. Das Protokoll ist innerhalb von 14 Tagen allen Mitgliedern zugänglich zu machen und muss auf der nächsten Mitgliederversammlung genehmigt werden.
\end{enumerate}


\section{Zuständigkeiten des Vorstands}
\begin{enumerate}
  \item Der Vorstand führt die Geschäfte des Vereins und fasst die erforderlichen Beschlüsse.
  \item Der Vorstand ist zu rechtsgeschäftlichen Verpflichtungen zu Lasten des Vereins bis zu einer Höhe von 500 Euro ermächtigt.
  \item Dem Vorstand obliegt insbesondere die Führung von Aufzeichnungen über Ausgaben und Einnahmen des Vereins.   Dazu kann vom Vorstand ein Vorstandsmitglied gewählt werden.
  \item In dringenden, keinen Aufschub duldenden Anliegen kann der Vorstand über diese Befugnisse hinaus handeln. Der Vorstand ist verpflichtet die Mitglieder hierüber unverzüglich zu informieren. Auf Verlangen eines Viertels der Mitglieder ist danach eine außerordentliche Mitgliederversammlung einzuberufen.
\end{enumerate}


\section{Zuständigkeiten der Mitgliederversammlung}
\begin{enumerate}
  \item Die Mitgliederversammlung:
  \begin{enumerate}
    \item wählt und kontrolliert den Vorstand;
    \item erteilt die Entlastung des Vorstandes;
    \item beschließt über Rechtsgeschäfte über 500 Euro;
    \item entscheidet in allen Fällen, in denen nicht die Zuständigkeit eines anderen Organs bestimmt ist;
    \item trifft Mehrheitsentscheidungen mit der Hälfte der teilnehmenden Mitglieder und einem Viertel der stimmberechtigten Mitglieder;
    \item wählt aus ihren Reihen einen Protokollführer, der den Ablauf der Mitgliederversammlung schriftlich protokolliert;
    \item kann sich eine Geschäftsordnung geben.
  \end{enumerate}
\end{enumerate}


\section{Ausschluss eines Mitgliedes}
\label{sec:ausschluss}
\begin{enumerate}
  \item Ein Mitglied kann aufgrund groben Fehlverhaltens wieder der Vereinsziele durch den Vorstandes aus dem Verein ausgeschlossen werden.
  \item Der Vorstand kann mit einfacher Mehrheit ein Mitglied auf Antrag ausschließen.
  \item Gegen den Ausschluss kann innerhalb eines Monats Widerspruch eingelegt werden.
  \item Ein Widerspruch führt zu einer Überprüfung des Ausschlusses durch die Mitgliederversammlung. Die einfache Mehrheit muss den Ausschluss bestätigen.
  \item Bis zu Entscheidung der Mitgliederversammlung ruht die Mitgliedschaft.
\end{enumerate}


\section{Auflösung}
\begin{enumerate}
  \item Zur Auflösung des Vereins bedarf es der Dreiviertelmehrheit der an der Mitgliederversammlung teilnehmenden Mitglieder und der Hälfte der stimmberechtigten Mitglieder.
  \item Bei Auflösung des Vereins oder bei Wegfall steuerbegünstigter Zwecke fällt das Vermögen des Vereins an den "`Förderverein Freie Netzwerke e.V."', der es unmittelbar und ausschließlich zu gemeinnützigen Zwecken zu verwenden hat.
  \item Im Auflösungsfall ist ein Liquidator zu bestellen.
\end{enumerate}


\section{Sonstiges}
\begin{enumerate}
  \item Beschlüsse, durch die eine für steuerliche Vergünstigungen wesentliche Satzungsbestimmung geändert, ergänzt, in die Satzung eingefügt oder aufgehoben wird oder die Auflösung des Vereins, die Überführung in eine andere Körperschaft oder die Übertragung des Vereinsvermögens als Ganzes, sind der zuständigen Finanzbehörde durch den Vorstand unverzüglich mitzuteilen.
  \item Vor der Verteilung oder Übertragung des Vereinsvermögens ist die Unbedenklichkeitserklärung des zuständigen Finanzamtes einzuholen.
\end{enumerate}


\section{Einblick in Vereinsdaten und Datenschutz}
\begin{enumerate}
  \item Der Vorstand erstellt eine Mitgliederliste, die auf Anfrage eines Mitglieds diesem ausgehändigt wird. Die Mitgliederliste enthält Name, Postanschrift, E-Mail-Adresse und Telefonnummer der Vereinsmitglieder.
  \item Die Mitgliederversammlung kann beschließen, dass die Mitgliederliste auch an bestimmte andere Personen weitergegeben wird.
  \item Jedes Mitglied kann verlangen, dass es nicht in der Mitgliederliste aufgeführt wird und dass seine Angaben anderen Personen nicht zugänglich gemacht werden.
  \item Jedes Mitglied kann Einblick in sämtliche Vereinsunterlagen verlangen.
\end{enumerate}


\section{Inkrafttreten}
Die Satzung tritt mit Gründung des Vereins in Kraft.

\end{document}
