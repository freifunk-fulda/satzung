\documentclass[ngerman]{article}

\usepackage[margin=1.4in]{geometry}
\usepackage[T1]{fontenc}
\usepackage[utf8]{inputenc}
\usepackage{textcomp}
\usepackage{graphicx}
\usepackage{babel}
\usepackage{titlesec}

\titleformat{\section}{\normalfont\Large\bfseries}{\S\thesection}{1em}{}

\title{Satzung\\
des Vereins \emph{Magrathea Laboratories}}

\date{Version 0.03\\
\today}

\begin{document}

\maketitle

\begin{center}
\emph{\includegraphics[scale=0.3]{logo}}
\par\end{center}

\thispagestyle{empty}
\pagebreak


\section{Preamble}
Die Informationsgesellschaft unserer Tage ist ohne Computer nicht mehr denkbar. Informations- und Kommunikationstechnologien verändern stetig das Verhältnis zwischen Mensch und Maschine und der Menschen untereinander. Trotz immer neuer und schnellerer Kommunikationsformen die eine  räumliche Nähe unnötig erscheinen lassen, kann der persönliche Austausch von Menschen mit gemeinsamen Interessen an Wissenschaft, Technologie und digitaler Kunst durch nichts ersetzt werden. Das Ausloten technischer Möglichkeiten, die Durchführung von Experimenten und die Abschätzung möglicher Auswirkungen neuer Technologien haben Tradition und sind Ausgangsbasis für jede Art von Forschung und Weiterentwicklung.

Das Streben dieser Vereinigung liegt in der Förderung aufgeklärten und selbständigen Denkens, Handelns und Kommunizierens, mittels moderner Informationssysteme; das Verständnis deren zugrunde liegender Techniken, um die gesamtgesellschaftlichen Implikationen dieser verstehen und beeinflussen zu können; als auch die Entfaltung von Kreativität und des menschlichen Forscherdrangs, um Kunst, Kultur und Wissenschaft voranzutreiben, wodurch die Etablierung eines globalen freien Zugangs zu Wissen gelingen soll und die Entwicklung der Menschheit, hin zu einer freien, toleranten Welt begünstigt wird.

\section{Name, Sitz, Rechtsfähigkeit, Geschäftsjahr}
\begin{enumerate}
\item Der Verein führt den Namen \emph{Magrathea Laboratories}
\item Sitz des Vereins ist Fulda.
\item Der Verein \emph{Magrathea Laboratories} soll in das Vereinsregister des zuständigen Amtsgerichtes eingetragen werden; nach der Eintragung führt er den
Zusatz \emph{e.V.}.
\item Geschäftsjahr des Vereins ist das Kalenderjahr.
\end{enumerate}

\section{Zweck des Vereins}
\begin{enumerate}
\item \emph{Magrathea Laboratories} verfolgt ausschließlich und unmittelbar gemeinnützige Zwecke im Sinne des Abschnitts ``Steuerbegünstigte Zwecke'' der Abgabenordnung 1977. 
\item Zweck des Vereins ist die Förderung von Wissenschaft und Forschung sowie die Jugend- und Erwachsenenbildung an der Hochschule Fulda und in der Region Fulda auf dem Gebiet der Computersicherheit, des Datenschutzes und des kreativen Umgangs mit neuen Technologien.
\item Der Satzungszweck wird insbesondere verwirklicht durch:
\begin{enumerate}
\item Schaffung und Bereitstellung einer für alle nachfolgend genannten Punkte förderlichen Infrastruktur;
\item Lernen durch Lehren als zentrales Weiterbildungselement. Im Mittelpunkt steht das Schaffen einer Umgebung, welche den selbstständigen Erwerb von Wissen und die Entwicklung der Fähigkeiten zur Wissensvermittlung fördert;
\item Schaffung eines modernen Datenschutzbewusstseins. Unter anderem durch öffentliche Vorträge und Diskussionsrunden zu gesellschaftspolitischen Fragestellungen im Hinblick auf das Recht zur informationellen Selbstbestimmung;
\item Durchführung von Fortbildungsveranstaltungen für Computersicherheit, Datenschutz und kreativen und kritischen Umgang mit neuen Technologien und deren
Anwendungen;
\item Ausbildung von Studierenden und Interessierten der Region Fulda auf dem Gebiet der Computersicherheit, des Datenschutzes und des kreativen Umgangs mit neuen Technologien und deren Anwendungen.
\end{enumerate}
\end{enumerate}

\section{Selbstlosigkeit}
\begin{enumerate}
\item Der Verein ist selbstlos tätig, er verfolgt nicht in erster Linie eigenwirtschaftliche Zwecke.
\item Mittel des Vereins dürfen nur für die satzungsmäßigen Zwecke verwendet werden. Die Mitglieder erhalten keine Zuwendungen aus Mitteln des
Vereins.
\item Es darf keine Person durch Ausgaben, die dem Zweck der Körperschaft fremd sind, oder durch unverhältnismäßig hohe Vergütungen begünstigt
werden.
\end{enumerate}

\section{Mitglieder}
\begin{enumerate}
\item Mitglieder können natürliche Personen und juristische Personen jedweder Rechtsform werden.
\end{enumerate}

\section{Beginn und Ende der Mitgliedschaft}
\begin{enumerate}
\item Der Vorstand entscheidet auf schriftlichen oder elektronischen Antrag des Antragstellers über die Aufnahme. Der Beschluss wird dem Antragsteller schriftlich oder elektronisch mitgeteilt. Drei reguläre Vorstandsmitglieder können vorübergehend einen Antrag annehmen.
\item Die Mitgliedschaft dauert mindestens 1 Jahr, danach verlängert sie sich jeweils um ein Jahr.
\item Die Mitgliedschaft endet:

\begin{enumerate}
\item bei juristischen Personen mit deren Auflösung.
\item bei natürlichen Personen mit ihrem Tod.
\item nach schriftlicher Kündigung eines Mitgliedes zum Ende des Mitgliedszeitraums nach Absatz 2. Die Kündigung muss mindestens 14 Tage vor Ablauf des Mitgliedszeitraums schriftlich oder elektronisch beim Vorstand eingegangen sein.
\item bei Mitgliedern, die sich trotz schriftlicher Mahnung mit mehr als einem Jahresbeitrag im Verzug befinden.
\item bei Ausschluss des Mitgliedes.
\end{enumerate}
\end{enumerate}

\section{Mitgliedsbeiträge}
\begin{enumerate}
\item Mitglieder entrichten, wenn die Mitgliederversammlung des Vereins das beschließt, Mitgliedsbeiträge in einer dann von ihr festgelegten Höhe. Der Beitrag wird dann im Voraus fällig.
\item An die Stelle der Mitgliedsbeiträge können mit Genehmigung des Vorstandes andere gleichwertige Zuwendungen treten.
\end{enumerate}

\section{Organe}
\begin{enumerate}
\item Die Organe von \emph{Magrathea Laboratories} sind die Mitgliederversammlung und der Vorstand.
\end{enumerate}

\section{Mitgliederversammlung}
\begin{enumerate}
\item Die Mitgliederversammlung besteht aus den Mitgliedern von \emph{Magrathea Laboratories}.
\item Die ordentliche Mitgliederversammlung wird einmal jährlich vom Vorstand einberufen.
\item Es kann eine außerordentliche Mitgliederversammlung einberufen werden. Dazu ist entweder ein Beschluss des Vorstandes oder die Zustimmung
von einem Drittel der stimmberechtigten Mitglieder notwendig.
\item Die Einladung zur Mitgliederversammlung ist den Mitgliedern elektronisch oder schriftlich unter Angabe von Ort, Zeit und Tagesordnung mindestens zwei Wochen im Voraus zuzustellen.
\item Anträge von Mitgliedern, die zusätzlich auf die Tagesordnung gesetzt werden sollen, müssen eine Woche vor dem Termin der Mitgliederversammlung elektronisch an alle Mitglieder geschickt werden. Dazu genügt das Versenden an die Vereinsmitgliedermailingliste, in die alle Vereinsmitglieder eingetragen werden.
\item Eine Vertretung eines Mitgliedes durch ein anderes Mitglied ist möglich, wenn die Vertretungsbefugnis schriftlich nachgewiesen werden kann oder unstrittig ist.
\item An einer Mitgliederversammlung kann elektronisch teilgenommen werden, wenn geeignete Techniken vorhanden sind.
\end{enumerate}

\section{Zuständigkeiten der Mitgliederversammlung}
Die Mitgliederversammlung:
\begin{enumerate}
\item wählt und kontrolliert den Vorstand,
\item prüft und genehmigt die Jahresabschlussrechnung des Vorstandes und erteilt die Entlastung,
\item beschließt über Rechtsgeschäfte über 1000 Euro,
\item entscheidet in allen Fällen, in denen nicht die Zuständigkeit eines anderen Organs bestimmt ist,
\item beschließt mit Dreiviertelmehrheit der Zahl der teilnehmenden Mitglieder und mindestens der Hälfte der stimmberechtigten Mitglieder über Auflösung oder Kooperation mit anderen Körperschaften,
\item trifft Mehrheitsentscheidungen mit der Hälfte der teilnehmenden Mitglieder und einem Zehntel der stimmberechtigten Mitglieder,
\item wählt aus ihren Reihen einen Protokollführer, der den Ablauf der Mitgliederversammlung schriftlich protokolliert,
\item kann sich eine Geschäftsordnung geben.
\end{enumerate}

\section{Vorstand}
\begin{enumerate}
\item Der Vorstand gemäß § 26 BGB besteht aus 5 Personen, die reguläre Mitglieder sind und wird auf 1 Jahr durch die Mitgliederversammlung gewählt. Eine Wiederwahl ist möglich.
\item Diese Mitglieder vertreten den Verein gerichtlich und außergerichtlich. Je 2 dieser Vorstandsmitglieder zusammen sind vertretungsberechtigt.
\item Darüber hinaus ist jeweils ein zusätzlicher Vorstandssitz mit einem Hochschullehrer der Hochschule Fulda als Vertreter der Wissenschaft
und Forschung und einem wissenschaftlichen Mitarbeiter als Vertreter des wissenschaftlichen Nachwuchses zu besetzten. Gelingt dies nicht,
bleibt der entsprechende Sitz unbesetzt.
\item Zu Sitzungen des Vorstandes ist eine Woche vorher schriftlich oder elektronisch zu laden. Mit dem Einverständnis aller Mitglieder des
Vorstandes kann diese Frist verkürzt werden oder ganz entfallen.
\item Der Vorstand ist beschlussfähig, wenn die Hälfte plus eines der Mitglieder des Vorstandes anwesend sind.
\item Beschlüsse im Vorstand werden mit einfacher Mehrheit gefasst.
\item Der Vorstand ist ermächtigt, gerichtlich oder behördlich geforderte Satzungsänderungen bis zur nächsten Mitgliederversammlung durchzuführen und umzusetzen.
\end{enumerate}

\section{Zuständigkeiten des Vorstands}
\begin{enumerate}
\item Der Vorstand führt die Geschäfte des Vereins und fasst die erforderlichen Beschlüsse.
\item Der Vorstand ist zu rechtsgeschäftlichen Verpflichtungen zu Lasten des Vereins bis zu einer Höhe von 1000 Euro ermächtigt.
\item Dem Vorstand obliegt insbesondere die Führung von Aufzeichnungen über Ausgaben und Einnahmen des Vereins. Dazu kann vom Vorstand ein Vorstandsmitglied gewählt werden.
\item In dringenden, keinen Aufschub duldenden Dingen kann der Vorstand über diese Befugnisse hinaus handeln. Der Vorstand ist verpflichtet
die Mitglieder hierüber unverzüglich zu informieren. Auf Verlangen von 10\% der Mitglieder ist danach eine außerordentliche Mitgliederversammlung einzuberufen.
\end{enumerate}

\section{Ausschluss eines Mitgliedes}
\begin{enumerate}
\item Der Vorstand kann mit einfacher Mehrheit ein Mitglied auf Antrag ausschließen.
\item Gegen diesen Ausschluss kann Widerspruch eingelegt werden.
\item Ein Widerspruch führt zu einer Überprüfung des Ausschlusses durch die Mitgliederversammlung. Die einfache Mehrheit muss den Ausschluss
bestätigen.
\item Bis zu Entscheidung der Mitgliederversammlung ruht die Mitgliedschaft.
\end{enumerate}

\section{Auflösung}
\begin{enumerate}
\item Zur Auflösung des Vereins bedarf es der Dreiviertelmehrheit der an der Mitgliederversammlung teilnehmenden Mitglieder und der Hälfte
der stimmberechtigten Mitglieder.
\item Bei Auflösung des Vereins oder bei Wegfall seines bisherigen Zwecks fällt das Vermögen von \emph{Magrathea Laboratories} an die Hochschule Fulda zwecks
Verwendung für die Förderung der Wissenschaft und Forschung.
\item Im Auflösungsfall ist ein Liquidator zu bestellen.
\end{enumerate}

\section{Sonstiges}
\begin{enumerate}
\item Beschlüsse, durch die eine für steuerliche Vergünstigungen wesentliche Satzungsbestimmung geändert, ergänzt, in die Satzung eingefügt oder aufgehoben wird oder die Auflösung des Vereins, die Überführung in eine andere Körperschaft oder die Übertragung des Vereinsvermögens
als Ganzes sind der zuständigen Finanzbehörde durch den Vorstand unverzüglich mitzuteilen.
\item Vor der Verteilung oder Übertragung des Vereinsvermögens ist die Unbedenklichkeitserklärung des zuständigen Finanzamtes einzuholen.
\end{enumerate}

\section{Inkrafttreten}

Die Satzung tritt mit Gründung des Vereins in Kraft.
\end{document}
